\section{La mise en œuvre technique du projet}

\subsection{L'organisation du projet}

Afin de travailler dans les meilleures conditions possibles, nous avons utilisé le gestionnaire de version \textbf{git} qui permet de gérer de manière approfondie le versionnage du projet et de travailler simultanément plus facilement. le projet se trouve sur le dépôt public \url{https://github.com/Neressea/EnglishProject}

\subsection{Les attentes atteintes}


\subsection{Les problèmes rencontrés et les solutions appliquées}

	
\subsection{Ce qui aurait pu être fait}
\paragraph{Les comptes administrateurs}
Bien que déjà implémenté dans la base de données, nous n'avons pas eu le temps de mettre en place les privilèges accordés aux administrateurs (édition et suppression de leçons, suppression de comptes utilisateurs).

\paragraph{Le choix du type de question}
Nous aurions pu proposer à l'utilisateur de choisir le type de question (texte à trous, QCM, question directe) désirée au lieu de décider ce type en fonction du thème de la question (vocabulaire, grammaire, compréhension).

\paragraph{Droits utilisateurs}
En l'état, un utilisateur ayant posté une leçon ne peut pas l'éditer ultérieurement. Nous aurions pu offrir cette possibilité.

\paragraph{Des réponses plus flexibles}
Les réponses données pour les questions de compréhension doivent pour l'instant être complètement identiques à la réponse entrée par l'enseignant. Les leçons seraient plus agréables pour l'élève si elles pouvaient admettre des réponses différentes ("Obama" à la place de Barack Obama, ou même des réponses très différentes mais toutes deux justes : "The president of the United States" au lieu de "Barack Obama").