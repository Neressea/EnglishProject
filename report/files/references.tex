\section*{Références et autres sources d'inspiration}
\addcontentsline{toc}{section}{Références}

\url{http://blog.hikoweb.net/index.php?post/2011/11/06/Exemple-de-rapport-en-LaTeX} \\
\emph{Source pour le template \LaTeX ayant servi à la rédaction de ce rapport.} \\

\url{http://www.crazymonkeygames.com/Boxhead-2Play-Rooms.html} \\
\emph{Source d'inspiration du jeu.} \\

\url{https://wiki.libsdl.org/} \\
\emph{Wiki de la SDL 2.0.} \\

\url{http://gnurou.org/} \\
\emph{Nous nous sommes inspirés des algorithmes de ce site pour la gestion des événements.} \\

\url{http://jeux.developpez.com/tutoriels/sdl-2/guide-migration/} \\
\emph{Guide de migration de passage de la SDL 1.2 à la SDL 2.0. Ce site nous a été utile étant donné
que nous avions déjà programmé avec la SDL 1.2 auparavant.} \\

\url{https://creativecommons.org/licenses/by-nc-sa/3.0/fr/} \\
\emph{Licence s'appliquant au logiciel.} \\

\url{http://www.pioneervalleygames.com/free-resources.html} \\
\url{http://opengameart.org/} \\
\emph{Sources des sprites libres utilisées dans le jeu.} \\

\url{https://www.draw.io/} \\
\emph{Site qui a servi à la création de l'UML} \\

\url{https://app.smartsheet.com} \\
\emph{Site qui a servi à la création du Gantt} \\

\url{http://soundbible.com/} \\
\url{http://www.audiomicro.com/} \\
\url{http://incompetech.com/music/royalty-free/} \\
\emph{Sources des fichiers audio utilisés dans le jeu.} \\

%% SPRITES %%
