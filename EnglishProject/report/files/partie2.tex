\section{La mise en œuvre technique du projet}

\emph{Nous rappelons que le site web peut-être testé depuis le lien suivant : \url{http://1.anglofun-1291.appspot.com}}

\subsection{L'organisation du projet}

Afin de travailler dans les meilleures conditions possibles, nous avons utilisé le gestionnaire de version \textbf{git} qui permet de gérer de manière approfondie le versionnage du projet et de travailler simultanément plus facilement. Le projet se trouve sur le dépôt public \url{https://github.com/Neressea/EnglishProject}

\subsection{les technologies mises en pratique}

\paragraph{}
Afin de décrire au mieux les technologies employées dans notre projet, nous allons distinguer ses deux principales parties : la partie back-end et la partie front-end.

\paragraph{Côté back-end.}
Du côté back-end, nous avons utilisé du python afin de coder le serveur web à l'aide du framework fourni par le Google App Engine qui permet entre autre de gérer les routes ainsi que le contenu des requêtes, et qui permet d'utiliser efficacement l'Active Record dans la gestion de la base de données. \linebreak

Nous avons aussi utilisé le gestionnaire de modèles Jinja2 permettant de générer des pages HTML à partir de "modèles" contenant non seulement le code HTML, mais aussi du code python (la génération des pages HTML devient ainsi plus flexible). \linebreak

Enfin, nous avons créé la base de données selon le modèle relationnel suivant :
\begin{flushleft}

USER \linebreak
\begin{itemize}
	\item isAdmin : boolean \linebreak
	\item username : String \linebreak
	\item password : String \linebreak
	\item grade\_vocabulary : int \linebreak
	\item grade\_grammar : int \linebreak
	\item grade\_comprehension : int \linebreak
	\item lessons\_done : List(int) \linebreak
\end{itemize}
	
	
\emph{Nous pouvons noter que l'attribut isAdmin, bien que précisé dans la base de données, n'est pas utilisé dans le reste du code. En effet nous n'avons pas eu le temps d'ajouter les privilèges administrateurs au site web. \linebreak Le mot de passe quant à lui est hashé avec la fonction hmac qui prend en paramètre un sel afin de rendre le hashage plus sécurisé. Il est aussi important de noter que cette fonction de hashage prend toujours le même temps, peu importe les capacités physiques de la machine ce qui permet d'éviter les bruteforce. \linebreak Enfin,la liste des leçons faites est une liste des id des leçons.} \linebreak

LESSON \linebreak
\begin{itemize}
	\item created\_by : String \linebreak
	\item difficulty : String in [Easy, Medium, Hard] \linebreak
	\item title : String \linebreak
	\item created : Date \linebreak
	\item last\_modified : Date \linebreak
	\item dominant : String in [Vocabulary, Grammar, Comprehension, Balanced] \linebreak
\end{itemize}
\emph{La difficulté et le thème majeur de la leçon sont des chaînes de caractères dont les valeurs doivent être comprises dans une liste de valeur acceptée.} \linebreak
 
STORY \linebreak
\begin{itemize}
	\item id\_lesson : int \linebreak
	\item num\_story : int \linebreak
	\item title : String \linebreak
	\item type\_of\_story in [Text, Video] \linebreak
	\item text : String \linebreak
	\item questions\_vocabulary : List(String) \linebreak
	\item questions\_grammar : List(String) \linebreak
	\item questions\_comprehension : List(String)  \linebreak
\end{itemize}

\emph{Le numéro de l'histoire correspond au numéro de l'histoire dans la leçon. Le type correspond au type du contenu et ne peut être que du texte ou une vidéo. Les questions sont des listes de chaines. Chaque élément de la liste correspond à une question. La chaine est formatée de la manière suivante : "question|bonne\_réponse|mauvaise\_réponse|..."} \linebreak
\end{flushleft}

\paragraph{Côté front-end.}

Pour le front-end, nous avons utilisés les technologies classiques (CSS, HTML, javscript). Mais afin d'être plus efficace et de proposer une expérience plus poussée à l'utilisateur, nous avons utilisés des frameworks pour ces technologies. \linebreak 
Pour le CSS, nous avons utilisés le framework bootstrap qui nous a permis de réaliser un design responsive plus rapidement. Et étant donné toutes les modifications effectuées sur le DOM de la page HTML (génération du formulaire en javascript, ajout dans le formulaire lors des ajouts d'histoires / questions, changement lors de la preview, ainsi que filtres de recherches) nous avons décidé d'utiliser jQuery afin de manipuler les sélecteurs CSS plus facilement. \linebreak
Du côté du front-end, nous avons rencontré quelques problèmes dans l'affichage de la vidéo (lors de la preview ou de l'affichage de la leçon). En effet, une connaissance nous a fait remarquer, après utilisation de notre site web sur son smartphone, qu'il ne pouvait ni voir de vidéo en preview ni en uploader. Nous nous sommes alors rendu compte qu'il existe 4 types de liens Youtube : 1 pour le site classique, un pour le site en version mobile, un pour l'application smartphone et enfin un pour les iframe. Dans notre cas, nous transformions les liens du site classique en lien pour iframe par remplacement de string, mais nous ne prenions pas encore les deux autres liens en compte. Nous les avons donc ajoutés.

\subsection{Les objectifs atteints et ce qui aurait pu être fait}
Toutes les fonctionnalités initiales que nous désirions implémenter ont été développées. On peut cependant noter que certaines fonctionnalités que nous avons commencé à implémenter n'ont pas pu être terminées.

\paragraph{Les comptes administrateurs.}
Bien que déjà implémentés dans la base de données, nous n'avons pas eu le temps de mettre en place les privilèges accordés aux administrateurs (édition et suppression de leçons, suppression de comptes utilisateurs).

\paragraph{Droits utilisateurs.}
En l'état, un utilisateur ayant posté une leçon ne peut pas l'éditer ultérieurement. Il est cependant envisageable d'apporter cette possibilité à long terme.

\paragraph{Le choix du type de question.}
Nous aurions pu proposer à l'utilisateur de choisir le type de question (texte à trous, QCM, question directe) désirée au lieu de décider ce type en fonction du thème de la question (vocabulaire, grammaire, compréhension).

\paragraph{Un formulaire plus adaptatif.}
Dans le formulaire, nous permettons d'ajouter des champs, mais pas de les supprimer. Cette option pourrait être intéressante si l'utilisateur se trompait dans l'ajout d'une question ou d'une histoire.

\paragraph{Des réponses plus flexibles.}
Les réponses données pour les questions de compréhension doivent pour l'instant être complètement identiques à la réponse entrée par l'enseignant. Les leçons seraient plus agréables pour l'élève si elles pouvaient admettre des réponses différentes ("Obama" à la place de Barack Obama, ou même des réponses très différentes mais toutes deux justes : "The president of the United States" au lieu de "Barack Obama").

\paragraph{Une utilisation de la courbe d'oubli.}
Enfin, nous aurions pu mettre à escient la "courbe d'oubli" afin de réduire les notes de l'utilisateur au cours du temps, pour permettre à l'utilisateur de profiter de rappels afin d'améliorer l'efficacité de l'apprentissage sur le long terme.